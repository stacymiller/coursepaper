\documentclass[12pt,a4paper]{article}
\usepackage[utf8]{inputenc}
\usepackage[english, russian]{babel}
\usepackage[T2A]{fontenc}

\usepackage[intlimits]{amsmath}
\usepackage{amsfonts}
\usepackage{amssymb}
\usepackage{amsthm}

\usepackage[backend=bibtex,style=authortitle]{biblatex}
\addbibresource{biblio.bib}
%\usepackage[left=2cm,right=2cm,top=2cm,bottom=2cm]{geometry}

\newcommand{\R}{\ensuremath{\mathbb{R}}}
\title{Устранение экспоненциальной сложности оценки стоимости бермудского опциона}
\author{Анастасия Миллер}
\date{СПбГУ, 6${}^{\mbox{\small ый}}$ семестр,~~ 322 гр. \\ \today} 
\begin{document}
\maketitle
\section{Вступление}
\par В книге \cite{Glasserman2004} был предложен метод оценки американских опционов с конечным множеством дат погашения. Две оценки -- смещённая вверх и смещённая вниз -- получаются с помощью смоделированного дерева, которое вевтится при каждой возможности раннего погашения опциона. Оценки являются состоятельными (т.е. сходятся по вероятности к истинной цене опциона) и асимптотически несмещёнными.
\par Один из основных недостатков алгоритма -- его экспоненциальная сложность. Здесь же предлагается несколько подходов, которые заменят экспоненциальную сложность полиномиальной с одновременным увеличением <<случайности>> алгоритма.
\section{Общая идея алгоритма}
\par Начиная с некоторого момента $t_k$, когда общее число состояний достигнет некоторого $n$, мы перестанем генерировать дочерние вершины ко всем состояниям. В следующий момент времени, $t_{k+1}$, мы будем иметь всё так же $n$ состояний, а не $bn$. Этого можно достичь, если генерировать дочерние состояния не ко всем вершинам, а только к некоторым. К каким?
\subsection{Анализ распределения состояний с помощью гистограммы}
\par В том случае, когда состояние актива $S$ является числом в $\R ^1$, в качестве параметра $X$, распределение которого нас интересует, можно использовать  само $S$, иначе можно использовать $h(S)$. 
\par Деля интервал $\left[\min_{i\in 1:n} X_i ; \max_{i\in 1:n} X_i + \frac{1}{n}\right)$ на $k$ равных частей $\left[a_{k-1},a_k\right), a_0 = \min_{i\in 1:n} X_i, a_k = \max_{i\in 1:n} X_i$, мы можем определить частоты $f_k = \#\left\lbrace X_i \middle\vert X_i\in\left[a_{k-1},a_k\right)\right\rbrace$ попадания событий в различные части отрезка. Из состояний, сгруппированных на отрезке $\left[a_{k-1},a_k\right)$, мы также можем создать некоторый <<средний арифметический>> вектор, кооринаты которого будут являться средним арифметическим координат всех состояний, оказавшихся на данном отрезке, и уже для этого нового среднего состояния -- представителя отрезка -- генерировать дочерние вершины в количестве $n f_k$. Для всех состояний, оказавшихся в этом отрезке, дочерними вершинами будут являться все вершины, полученные от их представителя. Таким образом, количество рассматриваемых состояний не увеличится. С другой стороны, этот метод предполагает хранение в памяти всего дерева, а не только непосредственно обсчитываемой ветки, как это предполагалось в исходной работе \cite{Broadie1997}.
\subsection{Кластеризация состояний}

\nocite{*}
\printbibliography
\end{document}