\documentclass[a4paper,10pt]{article}

\pdfcompresslevel=9

\usepackage{amsmath,amsthm,amssymb}
\usepackage{mathtext}
%\usepackage{cmap}

\usepackage[T1,T2A]{fontenc}
\usepackage[utf8]{inputenc}
\usepackage[english,russian]{babel}

\usepackage[top=1in,bottom=1in,left=1in,right=1in]{geometry}
\usepackage{hyperref}

\title{Оценка американских опционов. Метод случайного дерева}
\author{Миллер Анастасия Александровна}
\date{СПбГУ, 5${}^{\mbox{\small ый}}$ семестр,~~ 322 гр. \\ \today}
\setlength\parindent{0pt}
\renewcommand{\thesubsection}{\arabic{section}.\arabic{subsection}}
\renewcommand{\thesection}{\arabic{section}}

\newtheorem{theorem}{Теорема}
\newcommand{\ev}{\mathsf{E}}
\bibliographystyle{plain}
\begin{document}
\maketitle
\tableofcontents
\setlength{\parskip}{0.5em}

\section{Описание задачи}
%\subsection{Основные понятия}
%\par \emph{Опцион} -- договор, по которому потенциальный покупатель или потенциальный продавец актива (товара, ценной бумаги) получает право, но не обязательство, совершить покупку или продажу данного актива по заранее оговорённой цене в определённые договором моменты в будущем или на протяжении определённого отрезка времени.
%\\ {\footnotesize \textbf{Пример}: есть 10 акций акционерной компании <<Берёзка>>, которые сегодня стоят 5\$ за штуку. Этими акциями (активом) владеет Вася. \emph{Опционом колл} на этот актив будет бумага, свидетельствующая, к примеру, что владелец опциона Петя имеет право купить эти 10 акций компании <<Берёзка>> по цене 7\$ за штуку 31 декабря 2013 года. В случае, если 31 декабря 2013 года акции компании <<Берёзка>> будут стоить на рынке 10\$ за штуку, Петя может придти к Васе и, предъявив свой опцион, купить эти акции по 7\$, заработав на разнице 30\$.}
%
%\par Опционы подразделяются по типу договора на опционы на покупку и опционы на продажу, по дате исполнения (моменты времени, когда можно потребовать исполнения обязательств, указанных в контракте) -- на европейские (могут быть исполнены только в день истечения контракта), бермудские (могут быть исполнены лишь в некоторые фиксированные даты, указанные в контракте) и американские (могут быть исполнены в любое время не позже момента истечения срока контракта).
%\\ {\footnotesize \textbf{Пример} (продолжение): сколько заплатил Петя, чтобы иметь такой контракт от Васи? Если мы считаем, что деньги не делаются из воздуха, то Петя должен был заплатить Васе 30\$ для того, чтобы в итоге количество денег на рынке не увеличилось.}
%\par Для опционов существует понятие \emph{цены исполнения опциона} -- цены, установленной в опционном контракте за актив, на который заключается опцион, и \emph{премии опциона}(\emph{цены опциона}) -- цены, которую платит покупатель опциона, чтобы получить опционный контракт. Какова же цена опциона? Сколько стоит заключение такого контракта? В основе всех математических моделей по расчёту цены опциона лежит идея эффективного рынка. Предполагается, что «справедливая» премия опциона соответствует его стоимости, при которой ни покупатель опциона, ни его продавец, в среднем не получают прибыли.
\subsection{Формулировка задачи}
\par Необходимо установить размер справедливой премии опциона до того момента, как этот опцион будет кому-то продан.
\paragraph{Обозначения и умолчания}
\par Будем строить модель на примере Бермудского опциона, который может быть исполнен в каждый из фиксированных моментов времени $t_1, \ldots t_m$. Мы также сузим класс решаемых нами задач до тех, в которых вся необходимая информация об активе, на который выписан рассматриваемый опцион, может быть представлена в виде Марковского процесса $X\left( t \right), t \in \left\lbrace t_i \right\rbrace_{i = 1}^m$ со значениями в $\mathbb{R}^d$. Для уменьшения объёма текста будем обозначать $X\left(t_i\right) \equiv X_i$. Положим также $h_i\left(x\right)$ -- размер выплаты по опциону в момент $t_i$ при том, что $x = X_i$ и опцион не был исполнен до этого, $V_i\left(x\right)$ -- стоимость опциона в момент $t_i$ при том, что $x = X_i$.
\par Нетрудно видеть, что
\begin{eqnarray}\label{eq:option-recursive}
	V_m\left(x\right) = h_m\left(x\right) \\
	V_{i-1}\left(x\right) = \max\left\lbrace h_{i-1}\left(x\right), \mathsf{E}\left[V_i\left(X_i\right)|X_{i-1}=x\right]\right\rbrace
\end{eqnarray} - на каждом шаге мы выбираем наиболее выгодное решение. Здесь нас интересует значение $V_0\left(X_0\right)$.

\section{Метод случайного дерева}
Вместо того, чтобы строить оценку, каким-либо образом стремящуюся к требуемому нами значению, мы построим две оценочные функции, оценивающие $V_n$ сверху и снизу. Пусть $\hat{V}_n\left(b\right)$ и $\hat{v}_n\left(b\right)$ -- такие оценки, зависящие от некоторого параметра $b$.
\par Метод случайного дерева основан на моделировании цепи $X_0, X_1, \ldots X_n$. Зафиксируем параметр ветвления $b$. Из исходного состояния $X_0$ смоделируем $b$ независимых следующих состояний $X_1^1, X_1^2, \ldots X_1^b$, все с условием $X_1$. Для каждого $X_1^i$ снова смоделируем $b$ независимых последующих состояний $X_2^{i1}, \ldots X_2^{ib}$. На $m$-ом шаге будем иметь $b^m$ состояний, и это и есть источник основного недостатка этого метода -- его экспоненциальной алгоритмической сложности.

\subsection{Оценка сверху}
\par Определим $\hat{V}_i^{j_1, j_2 \ldots j_i}$, вдохновляясь \ref{eq:option-recursive}. В последних вершинах (листьях) дерева зададим
\begin{equation}\label{eq:upper-terminal}
	\hat{V}_m^{j_1 \ldots j_m} = h_m\left(X_m^{j1 \ldots j_m}\right)
\end{equation}
Идя вверх по дереву, зададим
\begin{equation}\label{eq:upper-node}
	\hat{V}_i^{j_1 \ldots j_i} = \max \left\lbrace h_i \left( X_i^{j_1 \ldots j_i} \right), \frac{1}{b} \sum_{j = 1}^b \hat{V}_{i+1}^{j_1 \ldots j_i j}\right\rbrace 
\end{equation}
\par С помощью индукции можно доказать, что наша оценка уклоняется вверх в каждом узле
\begin{theorem}
	$\forall i \in 1:n$
	\begin{equation*}
	\mathsf{E}\left[\hat{V}_i^{j_1\ldots j_i}|X_i^{j_1\ldots j_i}\right] \geqslant V_i\left(X_i^{j_1\ldots j_i}\right)
	\end{equation*}
\end{theorem}
\begin{proof}
	\par В листьях дерева неравенство выполняется как равенство по определению.
	\par Докажем, что если утверждение теоремы выполняется на $i+1$ шаге, то оно выполняется и на $i$. По определению
	\begin{equation*}
	\ev\left[\hat{V}_i^{j_1\ldots j_i}|X_i^{j_1\ldots j_i}\right] = \ev\left[ \max\left\lbrace h_i\left(X_i^{j_1\cdots j_i}\right), \frac{1}{b}\sum_{j = 1}^b \hat{V}_{i+1}^{j_1 \ldots j_i j}\right\rbrace | X_i^{j_1\cdots j_i} \right]
	\end{equation*}
	с помощью неравенства Йенсена ($\varphi\left(\ev\left[X\right]\right) \leqslant \ev\left[\varphi(X)\right]$) это можно оценить
	\begin{equation*}
	\ev\left[ \max\left\lbrace h_i\left(X_i^{j_1\cdots j_i}\right), \frac{1}{b}\sum_{j = 1}^b \hat{V}_{i+1}^{j_1 \ldots j_i j}\right\rbrace | X_i^{j_1\cdots j_i} \right] \geqslant \max\left\lbrace h_i\left(X_i^{j_1\cdots j_i}\right), \ev\left[ \frac{1}{b}\sum_{j = 1}^b \hat{V}_{i+1}^{j_1 \cdots j_i j} | X_i^{j_1\cdots j_i} \right] \right\rbrace
	\end{equation*}
	в силу того, что $\forall \, j \in 1:b \quad X_{i+1}^{j_1\cdots j_i j}$ - независимые одинаково распределённые случайные величины (и их математическое ожидание одинаково), $\ev\left[ \frac{1}{b}\sum_{j = 1}^b \hat{V}_{i+1}^{j_1 \cdots j_i j} \right] = \frac{1}{b}\sum_{j = 1}^b \ev\hat{V}_{i+1}^{j_1 \cdots j_i j} = \ev\hat{V}_{i+1}^{j_1 \cdots j_i 1}$, а в силу индукционного предположения
	\begin{equation*}
	\max\left\lbrace h_i\left(X_i^{j_1\cdots j_i}\right), \ev\left[ \hat{V}_{i+1}^{j_1 \cdots j_i 1} | X_i^{j_1\cdots j_i} \right] \right\rbrace \geqslant \max \left\lbrace h_i\left(X_i^{j_1\cdots j_i}\right), V_i\left(X_i^{j_1\ldots j_i}\right) \right\rbrace
	\end{equation*}
	Таким образом, $\ev\left[\hat{V}_i^{j_1\ldots j_i}|X_i^{j_1\ldots j_i}\right] \geqslant \max \left\lbrace h_i\left(X_i^{j_1\cdots j_i}\right), V_i\left(X_i^{j_1\ldots j_i}\right) \right\rbrace$
\end{proof}
\par Мы также доказываем, что $\hat{V}_i^{j_1\ldots j_i}$ сходится по вероятности к $V_i\left(X_i^{j_1\ldots j_i}\right)$ при $b \to \infty$. В листьях дерева это очевидно ($\hat{V}_m^{j_1 \ldots j_m} = h_m\left(X_m^{j1 \ldots j_m}\right)$ по определению), на $i-1$ шаге цена удержания опциона $\frac{1}{b} \sum_{j = 1}^b \hat{V}_{i+1}^{j_1 \ldots j_i j}$ является средним арифметическим независимых одинаково распределённых случайных величин и сходится по закону больших чисел. Сходимость распространяется и на саму оценку в силу непрерывности операции взятия максимума. Используя тот факт, что $\forall \, a, c_1, c_2 \in \mathbb{R} \; |\max\left(a, c_1\right) - \max\left(a, c_2\right)| \leqslant |c_1 - c_2|$, мы получаем
	\begin{equation*}
	\left|\hat{V}_i^{j_1\cdots j_i} - V_i\left(X_i^{j_1 \cdots j_i}\right)\right| \leqslant \frac{1}{b}\sum_{j=1}^b\left|\hat{V}_{i+1}^{j_1 \cdots j_i j} - \ev\left[V_{i+1}\left(X_{i+1}^{j_1\cdots j_i j}\right)|X_{i+1}^{j_1\cdots j_i}\right]\right|
	\end{equation*}
	что позволяет нам вывести из сходимости на $i+1$ шаге сходимость на $i$ шаге. Подробнее в \cite{Broadie1997}
\par Более того, асимптотически наша оценка оказывается не сдвинутой вверх, т.е. $\mathsf{E}\hat{V}_0 \to V_0\left(X_0\right)$
\subsection{Оценка снизу}
\par Значения оценки сверху в каждый момент времени -- это выбор максимума из стоимости опциона при его немедленном исполнении и математического ожидания стоимости удержания опциона. Но стоимость удержания опциона рассчитывается, исходя из дочерних узлов дерева состояний актива,то есть оценка сверху рассчитывается, опираясь на информацию о будущем. Чтобы убрать ошибку, связанную с этим, нам необходимо отделить механизм принятия решения о исполнении/удержании опциона от значений, полученных после принятия решения об удержании опциона.
\par По сути, нам нужно оценить $\max\left\lbrace a, \ev Y \right\rbrace$ с помощью $b$ независимых одинаково распределённых реализаций случайной величины $Y$ для некоторой константы $a$ и случайной величины $Y$. Оценка $\max\left\lbrace a, \bar{Y}\right\rbrace$ (где $\bar{Y}$ -- среднее значение выборки) является оценкой сверху, так как $\ev\max\left\lbrace a, \bar{Y}\right\rbrace \geqslant \max\left\lbrace a, \ev\bar{Y}\right\rbrace = \max\left\lbrace a, \ev Y\right\rbrace$, что мы и использовали в построеии нашей оценки сверху.
\par Разделим наше множество реализаций $\left\lbrace Y_i \right\rbrace _{i=1}^b$ случайной величины $Y$ на два независимых подмножества и вычислим их средние значения $\bar{Y}_1$ и $\bar{Y}_2$. Если положить
	\begin{equation}
	\hat{v} = \left\lbrace
		\begin{array}{l l}
			a, & \, \text{если } \bar{Y}_1 \leqslant a \\
			\bar{Y}_2, & \, \text{иначе} 
		\end{array}\right.
	\end{equation}
	мы отделим процесс принятия решения о исполнении/удержании опциона от оценки его стоимости (за решение будет отвечать $\bar{Y}_1$, за оценку - $\bar{Y}_2$). При этом оценка $\hat{v}$ является оценкой снизу:
	\begin{equation}
		\ev\hat{v} = \mathsf{P}\left(\bar{Y}_1 \leqslant a\right)a + \left( 1 - \mathsf{P}\left(\bar{Y}_1 \leqslant a\right) \right)\ev Y \leqslant \max\left\lbrace a, \ev Y \right\rbrace
	\end{equation}
	
\section{Результаты и планы}
\par Таким образом, мы получили оценку сверху (\eqref{eq:upper-terminal},\eqref{eq:upper-node}) и снизу (\eqref{eq:lower-terminal},\eqref{eq:lower-node}) для справедливой цены Бермудского опциона. Нашей конечной целью является оценка Американского опциона. Оценка для Американского опциона может быть получена из этой путём увеличения $m$. Так как данный метод предполагает $m \leqslant 5$, увеличивать его в рамках одного шага нет смысла. Возможно, построение последовательности деревьев, где <<хвост>> дерева стягивается в новую вершину, с увеличением длины последовательности принесёт больший успех.
\nocite{*}
\bibliography{biblio}
\end{document}