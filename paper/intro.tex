\intro
Работа посвящена некоторым методам оценки стоимости Американских опционов. Опцион --- один из производных финансовых инструментов (вместе с фьючерсами, свопами, контрактами на разницу цен), активно использующихся на современных финансовых рынках. Интерес представляет оценивание <<справедливой>> цены опциона.

Более подробно, \emph{опцион} --- договор, по которому покупатель опциона (потенциальный покупатель или потенциальный продавец базового актива — товара, ценной бумаги) получает право, но не обязательство, совершить покупку или продажу данного актива (\emph{исполнить} опцион) по заранее оговорённой цене (\emph{цене страйк}) в определённый договором момент в будущем или на протяжении определённого отрезка времени. При этом продавец опциона несёт обязательство совершить ответную продажу или покупку актива в соответствии с условиями проданного опциона. Различают опционы на продажу актива и на его покупку. По количеству и типу моментов времени, в которые можно совершить оговорённую в контракте сделку, опционы подразделяются на Европейские (совершить сделку можно только в один указанный момент времени) и Американские (совершить сделку можно в любое время вплоть до указанного), а также более экзотические варианты (например, Бермудский, позволяющий совершить сделку в один из нескольких указанных моментов).

Опционный контракт предполагает возможность смены владельца, а исполнение опциона (совершение покупки или продажи, право на которую даётся в контракте) может принести выгоду по сравнению с аналогичной операцией на рынке, поэтому опционный контракт имеет собственную цену. Значение этой цены неочевидно, так как невозможно точно предсказать размер прибыли, которую получит владелец контракта.

В работе рассматриваются имитационные методы оценки Американского опциона. В работе \cite{Broadie1997} построены оценки на имитационных моделях, являющиеся оценками сверху и снизу цены Американского опциона. Оценки разработаны в основном для опционов с конечным числом дат исполнения, но доказано, что с ростом времени эти оценки сходятся к истинному значению цены Американского опциона. Однако с ростом времени вычислительная работа растёт экспоненциально, что делает невозможным применение этих оценок в исходном виде для подсчёта Американского опциона. Задачей дипломной работы было указать методы, позволяющие избежать экспоненциального роста вычислительной работы.

% Целью работы является анализ и упрощение (вычислительное) некоторых методов оценки стоимости Американских опционов. В частности, рассматривается метод случайных деревьев и его возможные улучшения. Основной задачей работы является применение методов решения интегральных уравнений с помощью ветвящихся процессов для древовидной модели вычисления цены опциона.