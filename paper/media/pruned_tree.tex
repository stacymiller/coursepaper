\documentclass{standalone}
%Russian-specific packages
%--------------------------------------
\usepackage[T2A]{fontenc}
\usepackage[utf8]{inputenc}
\usepackage[russian]{babel}
%--------------------------------------
\usepackage{tikz}
\usetikzlibrary{shapes,snakes}
\usepackage{systeme}
% \usepackage{pgfplots}
\usetikzlibrary{calc}
\pgfmathsetseed{\number\pdfrandomseed}
\begin{document}
\def\layersep{2.5cm}
\def\vertsep{1.5cm}
\def\gridp{17}
\begin{tikzpicture}[shorten >=1pt,draw=black!50,yscale=-0.90, xscale=0.65]
    \tikzstyle{vertex}=[circle,color=black,fill=black!50,inner sep=1,draw]
    \tikzstyle{selected vertex}=[circle,color=red,fill=red!50,inner sep=1,draw]
    
    \draw (9.5, 0) node[rectangle] (root) {$X_0$};

    \draw[loosely dotted,color=black!50] (root) -- (20,0) node[color=black] {$t_0$};
    \foreach \line in {1,...,6}{
        \draw[loosely dotted,color=black!50] (0,\line) -- (20,\line) node[color=black] {$t_\line$};
    }
    \foreach \line in {3,2,1}{
        \draw[loosely dotted,color=black!50] (0,{10-\line}) -- (20,{10-\line}) node[color=black] {$t_{m-\line}$};
    }

    \foreach \branch in {1,2,3}{
        \draw ({5.5+4*(\branch-1)}, 1) node[vertex] (1-\branch) {};
        \draw[->] (root) -- (1-\branch);
        \foreach \child in {1,2,3}{
            \draw ({5.5+4*(\branch-1)+1*(\child-2)}, 2) node[vertex] (2-\branch\child) {};
            \draw[->] (1-\branch) -- (2-\branch\child);
        }
    }

    \foreach \n in {1,2}{
        \pgfmathsetmacro{\b}{random(1,3)}
        \pgfmathsetmacro{\c}{random(1,3)}
        \draw node[selected vertex] (r1\n) at (2-\b\c) {};
        \foreach \branch in {1,2,3}{
            \draw ({(\n-1)*9+3*\branch-1},3) node[vertex] (3-\branch) {};
            \draw[->] (r1\n) -- (3-\branch);
            \foreach \child in {1,2,3}{
                \draw ({(\n-1)*9 + \child + 3*\branch - 3}, 4) node[vertex] (4-\branch\child) {};
                \draw[->] (3-\branch) -- (4-\branch\child);
            }
        }
    }

    \foreach \n in {1,2}{
        \pgfmathsetmacro{\b}{random(1,3)}
        \pgfmathsetmacro{\c}{random(1,3)}
        \draw node[selected vertex] (r2\n) at (4-\b\c) {};
        \foreach \branch in {1,2,3}{
            \draw ({(\n-1)*9+3*\branch-1},5) node[vertex] (5-\branch) {};
            \draw[->] (r2\n) -- (5-\branch);
            \foreach \child in {1,2,3}{
                \draw ({(\n-1)*9 + \child + 3*\branch - 3}, 6) node[vertex] (6-\branch\child) {};
                \draw[->] (5-\branch) -- (6-\branch\child);
            }
        }
    }

    \foreach \mid in {2,5,8,11,14,17}{
        \draw (\mid, 6.75) node {$\vdots$};
    }

    \foreach \n in {1,2}{
        \pgfmathsetmacro{\b}{random(1,9)}
        \draw ({(\n-1)*9+\b},7) node[selected vertex] (r3\n) {};
        \foreach \branch in {1,2,3}{
            \draw ({(\n-1)*9+3*\branch-1},8) node[vertex] (8-\branch) {};
            \draw[->] (r3\n) -- (8-\branch);
            \foreach \child in {1,2,3}{
                \draw ({(\n-1)*9 + \child + 3*\branch - 3}, 9) node[vertex] (9-\branch\child) {};
                \draw[->] (8-\branch) -- (9-\branch\child);
            }
        }
    }

    % \foreach \line in {1,3,5}{
    %     \draw[dashed] (\line-3) -- (20,\line) node {\line};
    % }
    % \draw[dashed] (8-3) -- (20,8) node {$m-2$};
    % \foreach \line in {2,4,6}{
    %     \draw[dashed] (\line-33) -- (20,\line) node {\line};
    % }
    % \draw[dashed] (9-33) -- (20,9) node {$m-1$};
    % \draw[dashed] (r32) -- (20,7) node {$m-3$};
    % Draw the input layer nodes
    % \foreach \name / \y in {1,...,3}
    %     \node[input neuron, pin=left:Вход \#\y] (I-\name) at (0,-\vertsep*\y) {};
    % \foreach \name / \y in {1,...,3}
    %     \draw node at (I-\name) {$a^1_\name$};
    

    % % Draw the hidden layer nodes
    % \foreach \name / \y in {1,2}
    %     \path[yshift=-\vertsep*0.5]
    %         node[hidden neuron, split vertically] (H-\name) at (\layersep,-\vertsep*\y) {\rotatebox{-90}{$z^2_\name$} \nodepart{lower} \rotatebox{-90}{$a^2_\name$}};
    % % \foreach \name / \y in {1,2}
    % %     \path[yshift=-\vertsep*0.5]
    % %         node[split vertically] at (H-\name) {};

    % % Draw the output layer node
    % \coordinate (H-mid) at ($(H-1)!0.5!(H-2)$);
    % \node[output neuron,pin={[pin edge={->}]right:Выход}, right of=H-mid] (O) {};

    % \foreach \source in {1,...,3}
    %     \foreach \dest in {1,2}
    %         \path (I-\source) edge (H-\dest);
    %         % \draw node[split vertically] at (I-\source) {\nodepart{lower} \rotatebox{-90}{$a^1_3$}};

    % \foreach \source in {1,2}
    %     \path (H-\source) edge (O);

    % % Annotate the layers
    % \node[annot,above of=I-1, node distance=1cm] (il) {Входной слой};
    % \node[annot,right of=il] (hl) {~  };
    % \node[annot,right of=hl] {Выходной слой};

    % % Define task
    % \foreach \source in {1,...,3}
    %     \foreach \dest in {1,2}
    %         \draw (I-\source) -- node[near start] {$w^2_{\dest\source}$} (H-\dest);
    % \foreach \source in {1,2}
    %     \draw (H-\source) -- node[near start] {$w^3_{1\source}$}  (O);

    % % \node[annot, above of=hl] {\huge\textbf{A}};

    % % \draw node at (I-1) {1};
    % % \draw[->] (I-3) -- node[near start] {1} (H-1);
    % % \draw[->] (I-1) -- node[near start] {2} (H-2);
    % % \draw[->] (H-1) -- node[near start] {3} (O);
    % % \draw[->] (H-2) -- node[near start] {4} (O);
    % % \draw node[split vertically] at (H-1) {\rotatebox{-90}{5} \nodepart{lower} };
    % % \draw node[split vertically] at (H-2) {\nodepart{lower} \rotatebox{-90}{6}};
    % \draw node[split vertically] at (O) {\rotatebox{-90}{$z^3_1$} \nodepart{lower} \rotatebox{-90}{$a^3_1$}};
\end{tikzpicture}
\end{document}