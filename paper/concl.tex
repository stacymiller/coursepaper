\conclusion
В работе был исследован метод оценки стоимости Американских опционов с помощью построения древовидной имитационной модели. Основой работы послужила гипотеза о примененимости схемы Неймана-Улама для оценки уравнений с операторами, не являющимися непрерывными (такими как $\max$), в связи с чем было проведено исследование возможности обобщения схемы.

В результате работы установлена теоретическая возможность вычисления произведения двух функций, одна из которых задана интегральным уравнением с оператором, не являющимся непрерывным, с помощью статистического моделирования ветвящегося марковского процесса. Исследована состоятельность оценок, аналогичных схеме Неймана-Улама, и экспериментально установлено отсутствие существенного уменьшения дисперсии за то же вычислительное время при применении таких оценок относительно классического алгоритма случайного дерева.

Реализован алгоритм оценки стоимости Бермудского опциона по случайным деревьям и все другие оценки, упомянутые в главах 3 и 4.