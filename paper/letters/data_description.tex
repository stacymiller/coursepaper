\documentclass{article}
\usepackage{graphicx}
\usepackage[usenames,dvipsnames,svgnames,table]{xcolor}

\usepackage[a4paper,
            mag=1000, includefoot,
            left=3cm, right=1.5cm, top=2cm, bottom=2cm, headsep=1cm, footskip=1cm]{geometry}
\usepackage[T2A]{fontenc}
\usepackage[utf8]{inputenc}
\usepackage[english,russian]{babel}

\usepackage[style=verbose-ibid]{biblatex}
\addbibresource{../biblio-u.bib}

\ifpdf\usepackage{epstopdf}\fi

\usepackage{multirow}
\usepackage{enumitem}

% Точка с запятой в качестве разделителя между номерами цитирований
% \setcitestyle{semicolon}

\graphicspath{ {../media/} }

\usepackage[ruled,vlined]{algorithm2e}

\usepackage[intlimits]{amsmath}
\usepackage{amsfonts}
\usepackage{amssymb}
\usepackage{amsthm}
\usepackage{mathrsfs}

\usepackage{hyperref}
\newtheorem{theorem}{Теорема}
\newcommand{\E}{\mathrm{E}}
\newcommand{\vfi}{\varphi}
\newcommand{\eps}{\varepsilon}
\newcommand{\prob}[1]{\mathrm{P}\left(#1\right)}
\newcommand{\R}{\ensuremath{\mathbb{R}}}
\newcommand{\Tau}{\ensuremath{\mathcal{T}}}
\newcommand{\GothB}{\mathfrak{B}}
\newcommand{\norm}[1]{\left\lVert#1\right\rVert}
\newcommand{\abs}[1]{\left\lvert#1\right\rvert}
\newcommand{\Vhat}{\hat{V}}
\newcommand{\vhat}{\hat{v}}
\newcommand{\maxset}[1]{\max\left\lbrace#1\right\rbrace}
\newcommand{\deltat}{\Delta t}
\DeclareMathOperator{\correlation}{cor}
\newcommand{\corr}[2]{\correlation\left(#1, #2\right)}
\DeclareMathOperator*{\argmax}{arg\,max}
\DeclareMathOperator*{\argmin}{arg\,min}
\DeclareMathOperator{\dd}{d}

\setlength{\parskip}{10pt}
\setlength{\parindent}{0pt}

\begin{document}

Данные по результатам измерений приведены в файле \texttt{data.csv}. Каждая строка -- результат отдельного эксперимента, про который указаны:

\begin{description}[align=right]
\item[\texttt{b}] параметр алгоритма оценки стоимости опциона по случайным деревьям $b = 10, 20, 50, 100, 150, 200$.
\item[\texttt{type}] тип используемых случайных чисел: \begin{description}[align=right]
	\item [\texttt{MC}] псевдослучайные,
	\item [\texttt{QMC}] квазислучайные (последовательность Холтона),
	\item [\texttt{RQMC}] рандомизированные квазислучайные (последовательность Холтона, рандомизация сдвигом).
\end{description}
\item [\texttt{halton\_dim}] размерность используемой последовательности Холтона, пропущена для псевдослучайных чисел.
\item [\texttt{group}] группа наблюдений, имеющих один и тот же случайный сдвиг (имеет смысл только для рандомизированных квазислучайных чисел).
\item [\texttt{est\_upper}] значение оценки.
\end{description}




\end{document}