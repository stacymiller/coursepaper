\documentclass[12pt,a4paper]{article}
\usepackage[utf8]{inputenc}
\usepackage[english, russian]{babel}
\usepackage{csquotes}
\usepackage[T2A]{fontenc}

\usepackage[intlimits]{amsmath}
\usepackage{amsfonts}
\usepackage{amssymb}
\usepackage{amsthm}

\usepackage{tikz}
\usetikzlibrary{graphs}

\begin{document}
\section{Про распределение}
Когда мы строим случайное дерево (<<рядами>>, т.е. дочерние вершины порождаются от всех родителей одновременно), на $k$-той итерации процесса у нас есть набор вершин $X = (x_1,\cdots,x_n) \in \mathbb{R}^n$. Итерация состоит в том, что от каждой родительской вершины из $X$ мы генерируем дочерние вершины. Более точно -- генерируем $b$ реализаций случайной величины с распределением, зависящим от родительской вершины (в моём случае -- $N\left( \underbrace{x_i\left(1+\mu\triangle t\right)}_{\mu_i}, \underbrace{x_i\sigma \sqrt{\triangle t}}_{\sigma_i}  \right)$). \\
Если обозначать как $x_i^j, i\in 1:n, j\in 1:b$ $j$-ю дочернюю вершину вершины $x_i$ ($j$-ю случайную величину с распределением $N\left(\mu_i, \sigma_i\right)$), а за $X_{new}$ обозначить множество всех дочерних вершин $\left\lbrace x_i^j \right\rbrace_{j\in 1:b, i\in 1:n}$, то правда ли, что случайная величина, наугад взятая из $X_{new}$, имеет распределение, являющееся равномерной смесью распределений $\sum_{i=1}^n \frac{1}{n} N\left(\mu_i, \sigma_i\right)$?
\section{Про квантили смеси распределений} % (fold)
Как находить квантили смеси нормальных распределений? Т.е. если $\mathcal{P} = \sum_{i=1}^n \frac{1}{n} N\left(\mu_i, \sigma_i\right)$, $\xi \in \mathcal{P}$, каков лучший метод для решения системы
\[
\left\lbrace \begin{matrix}
    &P(\xi < x_1) = \frac{1}{n} \\
    &\vdots\\
    &P(\xi < x_{m-1}) = \frac{m-1}{m}
\end{matrix} \right.
\]
Можно расписать подробнее:
\[
\left\lbrace \begin{matrix}
    &\int_{-\infty}^{x_1} \frac{1}{n}\sum_{i=1}^n p_{N(\mu_i,\sigma_i)} (t) dt = \frac{1}{n} \\
    &\vdots\\
    &\frac{1}{n}\sum_{i=1}^n \int_{-\infty}^{x_{m-1}} p_{N(\mu_i,\sigma_i)} (t) dt = \frac{m-1}{m}
\end{matrix} \right.
\]

Эта система не решается аналитически (обратная функция от нормальной функции распределения не выражается ничем хорошим, обратная функция от суммы также ничем внятным не является), но, возможно, вы слышали/видели/знаете, как кто-то решал это численными методами? Я сама придумала совсем немного (запоминать вычисленные при решении каждого уравнения системы значения функции как приближения к решению следующих уравнений). Может быть, эта задача уже где-то решалась? \\
% section _ (end)
\end{document}