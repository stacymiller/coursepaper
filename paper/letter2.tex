\documentclass{letter}
\usepackage[T2A]{fontenc}
\usepackage[utf8]{inputenc}
\usepackage[english,russian]{babel}

\usepackage[intlimits]{amsmath}
\usepackage{amsfonts}
\usepackage{amssymb}
\usepackage{amsthm}
\usepackage{mathtools}
\newcommand{\vfi}{\varphi}
\begin{document}
\begin{enumerate}
\item Верно ли, что метод последовательных приближений для интегрального уравнения выглядит как $$\vfi_n\left(x\right) = \int K\left(x, y, \vfi_{n-1}\left(y\right)\right)\vfi_{n-1}\left(y\right) \mu\left(dy\right) + f\left(x\right) \quad \mod \mu$$, а не как $$\vfi_n\left(x\right) = \int K\left(x, y, \vfi_{n-1}\left(y\right)\right)\vfi_n\left(y\right) \mu\left(dy\right) + f\left(x\right) \quad \mod \mu$$? Потому что иначе непонятно, как мы можем сокращённо записывать это как $\vfi = \mathcal{K}\vfi^{\left(b\right)} + f$.
\end{enumerate}
\end{document}