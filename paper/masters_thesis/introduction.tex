%!TEX root = thesis.tex
\intro
В работе рассмотрены основные подходы к оценке стоимости Американских опционов и приведено их сравнение между собой (как теоретическое, так и на численных примерах). Основной целью работы было исследование вопроса о применении метода квази Монте-Карло к задаче оценки стоимости Американского опциона.

Ниже в этом разделе разобрано понятие опциона (секция \ref{sec:intro:option_definition}) и описана формальная постановка задачи нахождения стоимости Американского опциона (секция \ref{sec:intro:option_price}). В главе \ref{cha:classic_approaches_to_option_pricing} приведены основные способы оценки этой стоимости. В главе \ref{cha:QMC_for_variance_reduction} даны основные сведения о методе квази Монте-Карло и рассмотрены особенности этого метода в приложении к финансовым задачам (в частности --- задаче оценки стоимости Американского опциона). В главе \ref{cha:tree_pruning_for_american_option} предложен новый подход к оценке стоимости Американского опциона, основывающийся на идее <<прореживания>> дерева состояний случайного процесса. Глава \ref{cha:numerical_results} содержит результаты численных экспериментов, подтверждающих или опровергающих высказанные в предыдущих главах гипотезы.
% В главе \ref{cha:classic_approaches_to_option_pricing} описана задача оценки Американских опционов (в секции \ref{sec:option_price}) и приведены основные методы её решения (случайные деревья, стохастические сетки и линейная регрессия, раздел \ref{sec:estimators}). 
% Глава \ref{cha:variance_reduction} содержит описание методов снижения дисперсии оценок, в частности, применение квази Монте-Карло. 
% В главе \ref{cha:quasi_monte_carlo} более подробно разобрана теория квази Монте-Карло и приведены обоснования для выбора размерности квазислучайной последовательности. В большинстве секций теоретические сведения подкреплены демонстрацией вычислений на конкретных примерах.

\section{Понятие опциона} % (fold)
\label{sec:intro:option_definition}

Опцион --- это широко распространённый вторичный (производный) финансовый инструмент. Опцион является контрактом между продавцом опциона и покупателем опциона о том, что покупатель имеет право, но не обязательство, купить (в случае опциона на покупку, call option) или продать (в случае опциона на продажу, put option) указанный в контракте базовый актив по заранее оговорённой цене (\emph{цене страйк}) в определённый контрактом момент в будущем или на протяжении определённого отрезка времени. Продавца опциона контракт обязует совершить ответную продажу (для опциона на покупку) или покупку (для опциона на продажу) в случае, если покупатель пожелает исполнить своё право. Реализация такой сделки называется \emph{исполнением опциона}.

Различают опционы европейского и американского типа. Опцион европейского типа выписывается на фиксированный момент времени в будущем, опцион американского типа --- на отрезок времени. Промежуточный вариант, когда опцион может быть исполнен только в определённые даты (например, в конце каждого квартала в течение года), часто называют Бермудским опционом.

Исполнение опциона может быть выгодно его владельцу (когда цена базового актива в контракте ниже текущей рыночной в случае опциона на покупку, когда цена базового актива выше текущей рыночной в случае опциона на продаже), поэтому опционный контракт сам по себе тоже имеет стоимость. Ищется стоимость опциона в модели эффективного рынка, то есть такая цена $V$, при которой ни продавец, ни покупатель опциона в среднем не получают прибыли.

% section intro:option_definition (end)

\section{Стоимость Американского опциона как случайная величина} % (fold)
\label{sec:intro:option_price}

В случае опциона европейского типа существует решение в замкнутой форме (модель Блэка-Шоулса \cite{Black1973} и её усовершенствования). Оценка Американского опциона является более сложной задачей.

Опцион определяется 
\begin{itemize}[noitemsep,topsep=0pt]
\item своим временем жизни $[0;T]$, 
\item базовым активом $X$ (под $X(t)$ будем подразумевать состояние актива в момент времени $t$, являющееся случайной величиной, под $S(t) = S(X(t))$ --- цену базового актива в момент $t$), на который выписан опцион (список возможных активов на территории Российской Федерации представлен в \cite{fsfr}), 
\item процессом $U(t),\;t\in{[0;T]}$, представляющим дисконтированное значение функции выплат (разницы между рыночной стоимостью базового актива и ценой страйк, оговорённой в контракте; значение функции выплат показывает выгоду, получаемую владельцем опциона при исполнении),
\item множеством $\Tau$ моментов времени, в которых возможно исполнить опцион.
\end{itemize} 
Будем также считать, что существует $h_t: U(t) = h_t\left(X(t)\right)$. Тогда для Американского опциона с функцией выплат $h_t\left(X_t\right)$ нахождение цены $V$ --- это задача оптимальной остановки (optimal stopping problem):
\begin{equation}\label{eq:optimal_stopping}
V = \max_{\tau} \E h_\tau\left(X_\tau\right).
\end{equation}

При дискретизации \eqref{eq:optimal_stopping} (принятии предположения о том, что $\Tau$ -- конечное множество $\left\lbrace t_i\right\rbrace_{i=0}^n \in \left[0;T\right], t_0 = 0, t_n = T$) задача обретает эквивалентную формулировку о нахождении $V_0\left(X_0\right)$ в системе
\begin{equation}\label{eq:option-recursive}\begin{aligned}
            V_m\left(x\right) &= h_m\left(x\right), \\
            V_{i-1}\left(x\right) &= \maxset{h_{i-1}\left(x\right),\E\left[V_i\left(X_i\right)|X_{i-1}=x\right]}.
\end{aligned}\end{equation}
Здесь $\forall i \in{0\mathbin{:}n}\quad h_i = h_{t_i}$, и такие обозначения будут использоваться и далее в тексте.

Далее в работе в основном будет использоваться формулировка \eqref{eq:option-recursive}. 

% section intro:option_price (end)