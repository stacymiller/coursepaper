\documentclass{article}

\usepackage[a4paper,
            mag=1000, includefoot,
            left=3cm, right=1.5cm, top=2cm, bottom=2cm, headsep=1cm, footskip=1cm
            % margin=0.1cm
            ]{geometry}
\usepackage[T2A]{fontenc}
\usepackage[utf8]{inputenc}
\usepackage[english,russian]{babel}
\ifpdf\usepackage{epstopdf}\fi

% Использовать полужирное начертание для векторов
\let\vec=\mathbf

% Включать подсекции в оглавление
\setcounter{tocdepth}{2}

\usepackage[defaultmono]{droidmono}
\usepackage[T2A]{fontenc}
% \usepackage{csquotes}

\usepackage[intlimits]{amsmath}
\usepackage{amsfonts}
\usepackage{amssymb}
\usepackage{amsthm}
\usepackage{mathtools}
\usepackage{nicefrac}

\usepackage{algorithm2e}
\usepackage{graphicx}
\graphicspath{ {media/} }
\usepackage{listings}
\usepackage{color}

\renewcommand*{\algorithmcfname}{Алгоритм}
\renewcommand*{\listalgorithmcfname}{Список алгоритмов}

\usepackage[fixlanguage]{babelbib}
\selectbiblanguage{russian}

% \usepackage{natbib}
% \usepackage[style=numeric]{biblatex}
% \addbibresource{biblio-u.bib}

\usepackage{hyperref}
\newtheorem{theorem}{Теорема}
\newcommand{\ev}{\mathsf{E}}
\newcommand{\vfi}{\varphi}
\newcommand{\prob}[1]{\mathsf{P}\left(#1\right)}
\newcommand{\R}{\ensuremath{\mathbb{R}}}
\newcommand{\Tau}{\ensuremath{\mathcal{T}}}
\newcommand{\GothB}{\mathfrak{B}}
\newcommand{\norm}[1]{\left\lVert#1\right\rVert}
\newcommand{\Vhat}{\hat{V}}
\newcommand{\vhat}{\hat{v}}
\newcommand{\maxset}[1]{\max\left\lbrace#1\right\rbrace}
\DeclareMathOperator*{\argmax}{arg\,max}
\DeclareMathOperator*{\argmin}{arg\,min}


\setlength\parindent{0pt}
\setlength\parskip{0.4em}
\widowpenalty=10000
\clubpenalty=10000


\usepackage{tikz}
\usetikzlibrary{arrows}
\usetikzlibrary{positioning}
\usetikzlibrary{graphs}

%----------------------------------------------------------------
\begin{document}
Добрый день, меня зовут Анастасия Миллер и моя бакалаврская работа --- об оценивании американскох опционов с помощью имитационных моделей.

Что такое опцион? Опцион --- это договор, по которому потенциальный покупатель актива получает право, но не обязанность, купить оговорённый актив по указанной в договоре цене в определённый момент в будущем или на протяжении определённого отрезка в будущем. Понятно, что если указанная в договоре цена оказывается ниже рыночной в те моменты, когда опционный договор может быть исполнен, владелец опциона может получить какую-то прибыль от исполнения опциона. Аналогично для потенциального продавца и цены выше рыночной. Следовательно, сам опцион тоже является ценной бумагой, имеющей определённую стоимость. Эта стоимость является объектом оценки в моей работе.

Справедливой ценой опциона --- ценой опциона в гипотезе эффективного рынка --- является максимум ожидаемой выручки от исполнения опциона. Здесь приведена формула для опциона колл, опциона на покупку актива. $S_\tau$ --- цена актива, на который выписан опцион, в момент времени $\tau$, $K$ --- цена страйк, $e^{-r\tau}$ --- дисконтирующий множитель.

В работе рассматриваются модели, в которых работает предположение о том, что опционный контракт может быть исполнен лишь в некотором конечном множестве моментов времени. Это предположение приводит к формулировке задачи динамического программирования, представленной на слайде. Здесь $V_i$ --- цена опциона на актив в состоянии $X_i$, который может быть исполнен в моменты от $t_i$ до $t_m$, и понятно, что эта цена --- это максимум из той выручки, что может быть получена при немедленном исполнении опциона, и цены, полученной, если опцион не будет исполнен в $i$-й момент.

В статье Броади 1997 года построены оценки на имитационных моделях, которые являются оценками сверху и оценками снизу для функции $V$. Эти оценки с ростом времени сходятся к истинному значению. В них математическое ожидание заменяется на среднее арифметическое всех или части оценок на следующем шаге алгоритма (и именно эта замена является источником смещения получаемых оценок). Таким образом, мы видим, что рекуррентное выражение содержит оператор, который связывает значение в $k$ момент времени с $b$ значениями в последующий момент времени. Именно это приводит нас к визуализации задачи в виде дерева состояний, представленного на рисунке.

Однако с ростом времени вычислительная работа растёт экспоненциально, что препятствует использованию этих оценок при подсчёте цены А.о. Задачей дипломной работы было указать методы, позволяющие избежать экспоненциального роста вычислительной работы. 

Известно, что одним из общих методов, позволяющих избежать экспоненциального роста вычислительной работы в разных вычислительных методах, является метод рандомизации. Поэтому было принято решение попробовать строить оценки этих оценок с помощью случайно  выбранных поддеревьев на деревьях, число вершин в которых растёт экспоненциально.

Подобный приём (случайный выбор поддеревьев) используется при решении нелинейных уравнений, в которых присутствует полиномиальная нелинейность. Имеется формальное сходство, следовательно, можно использовать для выбора поддерева механизм, использованный в методе решения нелинейных уравнений. Поскольку метод поколений приводит к экспоненциальному росту памяти, выбираем лексикографический метод отбора, что приводит нас к следующему механизму моделирования.

Задаём некоторую переходную плотность $p(x_k, x_{k+1}^1, \cdots, x_{k+1}^b)$ и вероятность обрыва $g_k$, плотность нормирована надлежащим образом (интеграл по носителю равен $1-g_k(X_k)$). В случае, если наступает событие обрыва, из данной вершины более не генерируются потомки, в ином --- генерируется $b$ потомков с плотностью $\nicefrac{p(x_k, x_{k+1}^1, \cdots, x_{k+1}^b)}{1 - g_k(X_k)}$. Выбор вероятностей обрыва позволяет контролировать объём дерева. Чем меньше  вероятность обрыва, тем меньше дисперсия оценки, но тем больше объём дерева и сходство с исходной оценкой.

Сравнительные результаты, показывающие наличие сходимости, представлены на слайде. Наличие сходимости относительно очевидно и доказано в работе, при числе ветвей дерева, стремящемся к бесконечности, и числе испытаний, стремящемся к бесконечности, получаемые оценки стремятся к истинному значению.

Графики на этом слайде показывают среднюю ошибку оценок по поддеревьям с различными вероятностями обрыва: с убывающей линейно, квадратично и как корень из отношения текущего шага к общему числу шагов, и по полному дереву. Видно, что дисперсия оценок, полученных за одно и то же число элементарных операций, примерно одинакова.

В работе были рассмотрены оценки стоимости а.о., которые строятся на базе имитационных моделей. Эти оценки с ростом времени требуют экспоненциально растущей вычислительной работы. Для преодоления этой проблемы была использована идея рандомизации. Была использована аналогия, возникающая между методами д.п., используемыми при построении этих оценок, и методами решения уравнений с полиномиальной нелинейностью. Показано на вычислительных примерах, что <<проклятье размерности>> может быть преодолено подходящим выбором ветвящегося процесса.

\end{document}