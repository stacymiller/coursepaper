\conclusion

В работе были рассмотрены оценки стоимости американского опциона, которые строятся на базе имитационных моделей. Эти оценки требуют экспоненциально растущей с ростом времени вычислительной работы, и для преодоления этого препятствия была использована идея рандомизации. При этом была использована аналогия, возникающая между методами динамического программирования, используемыми при построении оценок американского опциона, и методами решения интегральных уравнений с полиномиальной нелинейностью.

Алгоритмы, предложенные в главах 2 и 3, реализованы, и с их помощью на вычислительных примерах показано, что соответствующее <<проклятье размерности>> может быть преодолено с помощью подбора соответствующих констант вероятности обрыва траектории. Таким образом, получен новый результат относительно возможности приближённой оценки без экспоненциального роста вычислительной работы.